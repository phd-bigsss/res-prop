\usepackage[utf8]{inputenc}
% \usepackage[spanish,es-tabla]{babel}
% \usepackage[fixlanguage]{babelbib}
\usepackage{geometry}
\geometry{letterpaper,left=2cm,top=2cm, right=2cm}
\usepackage{times}           
% \usepackage{caption}
% \captionsetup[figure, table]{labelfont={bf},labelformat={default},labelsep=period,list=no}
% \captionsetup[figure]{list=no}
% \captionsetup[table]{list=no}
\usepackage{graphicx}
\usepackage{float}
\usepackage{booktabs}
\usepackage{longtable}
\usepackage{array}
\usepackage{multirow}
\usepackage{wrapfig}
\usepackage{float}
\usepackage{colortbl}
\usepackage{xcolor}
\usepackage{pdflscape}
\usepackage{tabu}
\usepackage{threeparttable}
\usepackage{pdfpages} %para pdf portada

\usepackage{fancyhdr}
\pagestyle{fancy}
\lhead{}
\chead{}
\rhead{}
\cfoot{\thepage} % get the page number 
\renewcommand{\headrulewidth}{0pt}
\renewcommand{\footrulewidth}{0pt}


% \fancypagestyle{plain}{%
%   % \fancyhf{}% Clear header/footer
%   % \fancyfoot[OR]{\textbf{\thepage}}%
%   % \fancyfoot[EL]{\textbf{\thepage}}%
%   \renewcommand{\headrulewidth}{0pt}%
% }


% fuente: https://stackoverflow.com/questions/45963505/coverpage-and-copyright-notice-before-title-in-r-bookdown
%\let\oldmaketitle\maketitle 
%\AtBeginDocument{\let\maketitle\relax}

% \renewcommand{\tablename}{Tabla}
% \ifxetex
%   \usepackage{polyglossia}
%   \setmainlanguage{spanish}
%   % Tabla en lugar de cuadro
%   \gappto\captionsspanish{\renewcommand{\tablename}{Tabla}
%           \renewcommand{\listtablename}{Índice de tablas}}
% \else
%   % \usepackage[spanish,es-tabla]{babel}
% \fi

